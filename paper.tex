% -*- coding: utf-8 -*-
\documentclass[nocover]{lizhechen}

\usepackage[utf8]{inputenc}
\usepackage[T1]{fontenc}
\usepackage{niceframe}
\usepackage{swrule}
\usepackage{float}
\usepackage{multirow}
\usepackage{diagbox}
\usepackage{listings}
\usepackage{ubuntu}
\usepackage{fancyhdr} %页眉设置

\usepackage{wallpaper} %使用wallpaper宏包
\CenterWallPaper{0.7}{baige.jpg}

\lstset{ %  
	extendedchars=false,            % Shutdown no-ASCII compatible  
	language=C++,                % choose the language of the code  
	%basicstyle=\normalsize\tt,    % the size of the fonts that are used for the code  
	basicstyle=\small\fontUbuntuMono,
	%tabsize=3,                            % sets default tabsize to 3 spaces  
	%numbers=left,                   % where to put the line-numbers  
	%numberstyle=\small,              % the size of the fonts that are used for the line-numbers  
	%stepnumber=1,                   % the step between two line-numbers. If it's 1 each line  
	% will be numbered  
	%numbersep=5pt,                  % how far the line-numbers are from the code   %  
	%keywordstyle=\color[rgb]{0,0,1},                % keywords  
	%commentstyle=\color[rgb]{0.133,0.545,0.133},    % comments  
	%stringstyle=\color[rgb]{0.627,0.126,0.941},      % strings  
	%backgroundcolor=\color{white}, % choose the background color. You must add \usepackage{color}  
	showspaces=false,               % show spaces adding particular underscores  
	showstringspaces=false,         % underline spaces within strings  
	showtabs=false,                 % show tabs within strings adding particular underscores  
	%frame=single,                 % adds a frame around the code  
	%captionpos=b,                   % sets the caption-position to bottom  
	breaklines=true,                % sets automatic line breaking  
	breakatwhitespace=false,        % sets if automatic breaks should only happen at whitespace  
	%title=\lstname,                 % show the filename of files included with \lstinputlisting;  
	% also try caption instead of title  
	mathescape=true,escapechar=?    % escape to latex with ?..?  
	escapeinside={\%*}{*)},         % if you want to add a comment within your code  
	%columns=fixed,                  % nice spacing  
	%morestring=[m]',                % strings  
	%morekeywords={%,...},%          % if you want to add more keywords to the set  
	%    break,case,catch,continue,elseif,else,end,for,function,global,%  
	%    if,otherwise,persistent,return,switch,try,while,...},%  
}


\begin{document}
	\chead{\includegraphics[scale=0.03]{logo.png}\ 北京师范大学白鸽青年志愿者协会 -- 项目总结报告}
	
	\begin{center}
		\ 
		\\[20ex]
		{\heiti\huge \textbf{北京师范大学白鸽青年志愿者协会}}\\[3ex]
		{\heiti\huge \textbf{2016~年秋季学期常规志愿服务项目}}\\[3ex]
		{\heiti\huge \textbf{期末总结报告}}
		\\[70ex]
		\begin{table}[H]
			\begin{center}
				\begin{tabular}{ccc}
					{\Large \textbf{项目名称:}} & & {\Large 图书馆项目} \\
					 &  & \\
					{\Large \textbf{报告时间:}} & & {\Large 2016.12.11} \\
					 &  & \\
					{\Large \textbf{编写成员:}} & & {\Large 李喆琛、宋研霏} \\
				\end{tabular}
			\end{center}
		\end{table}
		%{\UbuntuMono\Large \textbf{项目名称:}\ \ \ \ \ \ \ \ \ \ 图\ 书\ 馆\ 项\ 目 }\\[2ex]
		%{\UbuntuMono\Large \textbf{报告时间:}\ \ \ \ \ \ \ \ \ \ 2016.12.11}\\[2ex]
		%{\UbuntuMono\Large \textbf{编写成员:}\ \ \ \ \ \ \ \ \ \ \ \ \ \ 李\ 喆\ 琛\ \ \ \ }
	\end{center}
	
	\section{项目基本信息}
	\subsection{项目基本信息}
	
	\begin{table}[H]
		\begin{center}
		\begin{tabular}{|c|c|c|c|}
			\hline
			单位名称 & 校园服务中心 & 项目名称 &  图书馆项目 \\
			\hline
			项目成立时间 & 2012年 & 项目实施地点 & 北京师范大学图书馆 \\
			\hline
			项目负责人 & 李喆琛 & 年级 & 2015级 \\
			\hline
			联系电话 & 17888802353 & 电子邮箱 & 201411213015@mail.bnu.edu.cn \\
			\hline
		\end{tabular}
		\end{center}
	\end{table}
	
	\subsection{团队基本信息}
	
	\begin{table}[H]
		\begin{center}
			\begin{tabular}{|c|c|c|c|c|c|}
				\hline
				岗位 & 姓名 & 年级 & 联系电话 & 电子邮箱 & 主要分工\\
				\hline
				项目第二负责人 & 无 & \ & \ & \ & \ \\
				\hline
				项目财务负责人 & 无 & \ & \ & \ & \ \\
				\hline
				项目宣传负责人 & 无 & \ & \ & \ & \ \\
				\hline
				
			\end{tabular}
		\end{center}
	\end{table}
	
	\subsection{合作方面基本信息}
	\begin{table}[H]
		\begin{center}
			\begin{tabular}{|c|c|c|c|c|c|}
				\hline
				姓名 & 所属单位 & 职务 & 邮箱 & 电话 & 提供支持类型 \\
				\hline
				李永明 & 图书馆 & 分党委副书记 & liym@lib.bnu.edu.cn & 13522761697 & 志愿工作和场所安排 \\
				\hline
			\end{tabular}
		\end{center}
	\end{table}
	
	\section{项目实施}
	\subsection{项目开展活动详细描述}
	\begin{table}[H]
		\begin{center}
			\begin{tabular}{|c|c|c|c|c|}
				\hline
				时间 & 地点 & 活动名称 & 实施内容和步骤 & 参与人 \\
				\hline
				\multirow{6}{2cm}{$2016.9.19$} &  &  & 由图书馆分党委副书记李永 & \\
									& &  & 明老师为各位志愿者进行志 & \\
									& 图书馆 & 岗前 & 愿服务内容及要求的讲解;  & 图书馆全体 \\
									& 会议室 & 培训 & 项目负责人进行补充并介绍 & \\
									& &      & 白鸽概况以及志愿北京概况 & \\
									& & & &\\
				\hline
				\multirow{8}{2cm}{$2016.9.21$} &  &  & 主任首先介绍了白鸽概况; & \\
						 &  &  & 接下来校服各项目负责人自 & \\
						 & &   & 我介绍并讲述项目历史及概 & \\
						 & 教二 &    校服   & 况;大会中间,认证考评部 & 图书馆全体 \\
						 & 教室 &   全员    & 部长吴双讲解志愿北京注册 & \\
						 & &    大会   & 流程;大会末尾,踏鸽行项 & \\
						 & &       & 目进行了舞蹈串烧节目表演 & \\
						 & &       & 最后合影留念。 & \\
				\hline
				\multirow{2}{2cm}{$2016.10.19$} & 教二 & 图书馆 & 军训过后再次向志愿者 & 项目负责人、 \\
						 & 教室 & 例会 & 强调志愿服务内容要点 & 2016级全体\\
				\hline
				\multirow{2}{2cm}{$2016.10.17$} & 图书馆 & 试服务 & 2016级志愿者第一周上岗 & 图书馆全体志愿者 \\
							$\sim2016.10.23$ & & & &\\
				\hline
				\multirow{3}{2cm}{$2016.10.24$} & 教二 & 图书馆 & 由校园服务中心主任团对报 & 图书馆学干 \\
							& 教室 & 学干 & 名图书馆学干的2016级志愿 & 项目负责人\\
							& &面试  & 者进行筛选和面试 & \\
				\hline
				\multirow{2}{2cm}{$2016.12.3$} & 邱季端 & 公益月 & 和踏鸽行共办“书山有路鸽 & 图书馆外场 \\
							& 西侧 & 外场 & 为径”活动 & 人员 \\
				\hline
			\end{tabular}
		\end{center}
	\end{table}
	
	\subsection{项目取得的主要成果}
	\begin{itemize}
		\item 图书馆新增岗位“报刊管理助理岗位”;
		\item 读者服务岗的服务质量显著提高;
		\item 稳定运作。
	\end{itemize}
	
	\section{项目荣誉情况}
	\par 本学期项目获得的校内(包括白鸽)、校外奖项
	\begin{table}[H]
		\begin{center}
		\begin{tabular}{|c|c|}
			\hline
			奖项属性 & 奖项全名 \\
			\hline
			无 & 无 \\
			\hline
		\end{tabular}
		\end{center}
	\end{table}
	
	\section{财务报告}
	无
	
	\section{项目反馈评价}
	
	\subsection{项目成员评价}
	\begin{itemize}
		\item 项目优点
		\begin{enumerate}
			\item 工作环境安逸;
			\item 志愿服务时间和出勤情况可以较好地保证;
			\item 志愿者工作热情较高;
			\item 工作安排明确,工作中遇到的问题可以及时得到反映;
			\item 志愿者和项目负责人以及志愿者之间的沟通较好。
		\end{enumerate}
		
		\item 项目缺点
		\begin{enumerate}
			\item 项目知名度不够高;
			\item 一些意见和反馈虽然能够及时反映,但是不能得到及时解决;
			
		\end{enumerate}
		
		\item 项目改进方案
		\begin{enumerate}
			\item 在适当的时候举办活动,提高项目在校园中的知名度;
			\item 多做几篇推送,由白鸽公众号推送,提高项目的知名度;
			\item 加强志愿者和老师的沟通,提高解决问题的效率。
		\end{enumerate}
		
	\end{itemize}
	
	\subsection{项目主管单位评价}
	\par \ 
	\\[17ex]
	
	\subsection{会长团评价}
	\par \ 
	\\[20ex]
		
\end{document}
